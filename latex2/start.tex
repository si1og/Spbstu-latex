%!TEX program = lualatex
\documentclass[9pt,a5paper]{article}
\special{papersize=148mm,210mm}
\usepackage[left=1.5cm, right=1.5cm, top=1.5cm]{geometry}

\usepackage{graphicx}

% \usepackage[condensed,math]{iwona}
\usepackage[russian,english]{babel}
\usepackage[T2C]{fontenc}
\usepackage{fontspec}


% \usepackage{cmbright}
% \usepackage{unicode-math}
% \usepackage{polyglossia}

% \usepackage[no-math,no-config]{fontspec}
\setmainfont{f}[
	Extension = .ttf,
	UprightFont = *Light,
	ItalicFont = *LightItalic,
	BoldFont = *Bold,
	BoldItalicFont = *BoldItalic,
]


\usepackage{array}
\usepackage{tabularx}
\usepackage{ragged2e}
\usepackage{amsmath}

% Package for graphs
\usepackage{tikz}
\usepackage{tikz-network}
\usetikzlibrary{positioning}

% Package for letterspacing
\usepackage[letterspace=300]{microtype}

% \setdefaultlanguage{russian}

% For russian controls
\usepackage{amssymb}

\title{СИМПЛЕКТИЧЕСКИЕ МНОГООБРАЗИЯ}
\author{Илья Семенов}

\pagenumbering{gobble}

\newcommand{\verticalGap}{-.7cm}

% Makes possible using vars
\makeatletter

% Document margins correction
\sloppy

% Add new table column styles
\newcolumntype{C}[1]{>{\centering\let\newline\\\arraybackslash\hspace{0pt}}m{#1}}
\newcolumntype{R}[1]{>{\RaggedLeft\arraybackslash}p{#1}}

\setlength{\tabcolsep}{0pt}
\renewcommand{\arraystretch}{1}

% Replace eng >= <= to rus ones
\renewcommand{\leq}{\leqslant}
\renewcommand{\geq}{\geqslant}

% Remove space around align
\setlength{\abovedisplayskip}{0pt}
\setlength{\belowdisplayskip}{0pt}
\setlength{\abovedisplayshortskip}{0pt}
\setlength{\belowdisplayshortskip}{0pt}

% Set default latex font
\newcommand\Lcs[1]{{\normalfont\ttfamily\textbackslash#1}}

\begin{document}
	\begin{tabularx}{\textwidth}{m{.2\textwidth} C{.6\textwidth} R{.2\textwidth}}
		\footnotesize180& \footnotesize\@title & \footnotesize[ГЛ.8
	\end{tabularx}

	\vspace{10pt}

	\small\textbf{Д. Условие коммутативности потоков. } Пусть $\pmb{A}, \pmb{B}$ -- векторные поля на многообразии $M$.

	\small\textls{Теорема}. \textit{Два потока $A^{t}, B^{s}$ коммутируют тогда и только тогда, когда скобка Пуассона соответствующих векторных полей $\pmb{[A,\hspace{5pt} B]}$ равна нулю.}

	\small\textls{Доказательство.} Если $A^{t} B^{s} \equiv B^{s}A^{t}$ то по лемме 1 $\pmb{[A, B]} = 0$ Если $\pmb{[A, B]} = 0$ то по лемме 1 для любой функции $\varphi$ в любой точке \textit{x}
	\[
		\varphi(A^{t}B^{s}x) - \varphi(B^{s}A^{t}x) = o(s^2+t^2), \quad s \xrightarrow{} 0, t \xrightarrow{} 0
	\]
	\small Мы покажем, что отсюда вытекает $\varphi(A^{t}B^{s}x) = \varphi(B^{s}A^{t}x)$ при доста- точно малых $s$ и $t$.
	
	\footnotesize Применяя это соотнощение к локальным координатам $(\varphi = x_1,\,\dots,\,\varphi = x_n)$, получим $A^{t}B^{s} \equiv B^{s}A^{t}$,

	\footnotesize Рассомтрим прямоугольник $0 \leq t \leq t_0$, $0 \leq s \leq s_0$ (рис. 170) на плоскости $(t, s)$.
	Каждому пути, ведущему из $(0, 0)$ в $(t_0, t_0)$ и состоящему из конечного числа отрезков координатных направлений, сопоставим произведение
	преобразований потоков $A^{t}$ и $B^{s}$.

	\begin{tikzpicture}[scale=1, transform shape]
		% \draw [gray!50]  (0,0) -- (1,1) -- (3,1) -- (1,0)  -- (2,-1) -- cycle;
		% \draw [red] plot [smooth cycle] coordinates {(0,0) (1,1) (3,1) (1,0) (2,-1)};

		% \draw [gray!50, xshift=4cm]  (0,0) -- (1,1) -- (2,-2) -- (3,0);
		% \draw [cyan, xshift=4cm] plot [smooth, tension=2] coordinates { (0,0) (1,1) (2,-2) (3,0)};
		Внешняя фигура
		\draw [gray!50] (0,0) -- (-2.5,.1) -- (-3,3) -- (.5,3.05) -- cycle;
		\draw [red] (0,0) to[out=165,in=10] (-2.5,.1) to[out=82,in=-60] (-3,3) to[out=20,in=160] (.5,3.05) to[out=-92,in=75] (0,0);

		Внутренняя фигура
		\draw [gray!50] (0,2.3) -- (-.5,.5) -- (-2,.6) -- (-2.4,2.5) -- (-.6,2.9);
		\draw [red] (0,2.3) to[out=-100,in=70] (-.5,.5) to[out=172,in=4] (-2,.6) to[out=95,in=-65] (-2.4,2.5) to[out=25,in=178] (-.6,2.9);

		\Vertex[x=0, size=.6, color=none]{Au} 
		\Vertex[x=-2.5*2,y=.1*2, size=.6, color=none]{Bu} 
		\Vertex[x=-3*2,y=3*2, size=.6, color=none]{Cu} 
		\Vertex[x=.5*2,y=3.05*2, size=.6, color=none]{Du} 

		\Edge[lw=1pt,bend=-15](Au)(Bu)
		\Edge[lw=1pt,bend=-15](Bu)(Cu)
		\Edge[lw=1pt,bend=16](Cu)(Du)
		\Edge[lw=1pt,bend=5](Du)(Au)

		% \Vertex[x=0,y=2.3*2, size=.6, color=white]{Ain} 
		% \Vertex[x=-.5*2,y=.5*2, size=.6, color=none]{Bin} 
		% \Vertex[x=-2*2,y=.6*2, size=.6, color=none]{Cin} 
		% \Vertex[x=1,y=1, size=.6, color=none]{Din} 
		% \Vertex[x=1,y=1, size=.6, color=none]{Ein} 
		% \Vertex[x=1,y=1, size=.6, color=none]{Fin} 
		% \Vertex[x=1,y=1, size=.6, color=none]{Gin} 
		% \Vertex[x=1,y=1, size=.6, color=none]{Iin} 
		% \Vertex[x=1,y=1, size=.6, color=none]{Jin} 
		% \Vertex[x=1,y=1, size=.6, color=none]{Kin} 
		% \Vertex[x=1,y=1, size=.6, color=none]{Lin} 
		

	\end{tikzpicture}
\end{document}